
\documentclass{article}


%%%%%%%%%%%%%%%%%%%%%%%%%%%%%%%%%%%%%%%%%%%%%%%%%%%%%%%%%%%%
%%%%   表格 
%%%%
%\usepackage{colortbl} % 彩色表格
%\usepackage[table]{xcolor} % 自动加载 colortbl 为表格染色
%\usepackage{booktabs} % 可以使用三线表
\usepackage{multirow} % 复杂表格,使用multirow必须加载该package
%\usepackage{diagbox} % 有斜线表头的表格
%\usepackage{dcolumn} % 表格中小数点对齐


%%%%%%%%%%%%%%%%%%%%%%%%%%%%%%%%%%%%%%%%%%%%%%%%%%%%%%%%%%%%
%%%%   图片
%%%%
\usepackage{graphicx,subfigure,boxedminipage}
\usepackage{tikz}
\usepackage{overpic}
% 浮动体
\usepackage{float}

%图片所在的目录
\graphicspath {{figure/}} 

% 校徽logo
%\logo{\includegraphics[height=0.1\textwidth]{hfut_logo_red.pdf}} 
%\logo{\pgfuseimage{logo}}
%\pgfdeclareimage[height=5cm]{logo}{ustc_logo_new.pdf}


%%%%%%%%%%%%%%%%%%%%%%%%%%%%%%%%%%%%%%%%%%%%%%%%%%%%%%%%%%%%
%%%%   颜色
%%%%
\usepackage{xcolor}
%\usepackage{color}
%% beamer中已经定义的颜色:
%% red,green,blue,cyan,magenty,yellow,black,
%% darkgray,gray,lightgray,orange,violet,purple,brown
%% 自定义颜色:
%% \xdefinecolor{lanvendar}{rgb}{0.8,0.6,1}
%% \xdefinecolor{olive}{cmyk}{0.64,0,0.95,0.4}
%% \colorlet{structure}{blue!60!black}
% 自定义颜色,用“structure”表示 60%蓝色+40%黑色的颜色
\colorlet{structure}{blue!85!white}

%自定义背景颜色
\definecolor{bgColor_}{rgb}{0.8,0.85,0.85}

% 背景色, 上25%的蓝, 过渡到下白.
%\setbeamertemplate{background canvas}[vertical shading][bottom=white,top=structure.fg!25]
%\beamertemplateshadingbackground{white}{blue!25} %设置渐变(gradient)背景色,
%\beamersetaveragebackground{yellow!25} % 设置单一的(solid)背景色
%\beamertemplategridbackground[0.3cm] % 设置栅格(grid ) 背景

% 自定义命令. item 逐步显示时, 
% 使将要出现的item、正在显示的item、已经出现的item、 呈现不同颜色.
\def\hilite<#1>{\temporal<#1>{\color{blue!80}}{\color{blue!85!white}}{\color{black}}} % magenta

% 自定义命令,文体着色加粗
\newcommand{\renderTextBold}[2][blue]{\textbf{\color{#1}#2}}

% 自定义命令,段落背景色,可以包含引用
\newcommand{\cbox}[2][bgColor_]{\colorbox{#1}{\parbox{\linewidth}{\strut#2\strut}}}

% %%%%%%%%%%%%%%%%%%%%%%%%%%%%%%%%%%%%%%%%%%%%%%%%%%%%%%%%%%%%
% %%%%    主题
% %%%%
% \mode<presentation> {

%   % With navigation bar: default, boxes, Bergen, Madrid, Pittsburgh, Rochester
%   % With a treelike navigation bar: Antibes, JuanLesPins, Montpellier.
%   % With a TOC sidebar: Berkeley, PaloAlto, Goettingen, Marburg, Hannover
%   % With a mini frame navigation: Berlin, Ilmenau, Dresden, Darmstadt, Frankfurt, Singapore, Szeged
%   % With section and subsection titles: Copenhagen, Luebeck, Malmoe, Warsaw  
%   \usetheme{Antibes} %[hideothersubsections] beamer 模板的模式

%   % Inner color themes, 其他选择: orchid,albatross,beaver,beetle,default,crane,dolphin,dove,fly,orchid,lily,rose,seagull,seahorse
%   % Inner color themes, 其他选择: sidebartab,whale,wolverine
%   \usecolortheme{sidebartab}% default

%   % default,circles,margin,rounded,rectangles
%   \useinnertheme[shadow=true]{rounded} 

%   % default,infolines,miniframes,shadow,smoothbars,smoothtree,tree,sidebar,split
%   % \useoutertheme[height=0.1\textwidth,width=0.15\textwidth,hideothersubsections]{sidebar}

%   \setbeamercovered{dynamic} % dynamic,transparent,invisible

% }



%%%%%%%%%%%%%%%%%%%%%%%%%%%%%%%%%%%%%%%%%%%%%%%%%%%%%%%%%%%%
%%%%   字体
%%%%

% 英文默认字体都是 Times New Roman
%\usepackage{times} 
%\usepackage[english]{babel}

% 中文
\usepackage[fontset=windowsnew]{ctex}

\setmainfont{Times New Roman}
\setCJKmainfont{KaiTi}

%\usefonttheme[onlymath]{serif}
%\usefonttheme{serif} % Times New Rome 字体
%\setCJKmainfont{SimHei}
%\setCJKmainfont{FangSong}
%\setCJKmainfont{KaiTi}
%\setCJKmainfont{YouYuan}
%\setCJKmainfont{LiSu}
%\setCJKmainfont{Georgia}
%\setCJKsansfont{SimHei}
%\setCJKsansfont{FangSong}
%\setCJKsansfont{KaiTi}
%\setCJKsansfont{YouYuan}
%\setCJKsansfont{LiSu}
%\setCJKsansfont{Georgia}


%%%%%%%%%%%%%%%%%%%%%%%%%%%%%%%%%%%%%%%%%%%%%%%%%%%%%%%%%%%%
%%%%   其它
%%%%
\usepackage{enumerate}

% 抄录环境
\usepackage{alltt}
%\usepackage{verbatim}

% 数学符号
\usepackage{bm} %加粗版本的符号
\usepackage{amsmath,amssymb,amsfonts}

%超链接
\usepackage[colorlinks, linkcolor=red,anchorcolor=blue,citecolor=green]{hyperref} 

% 播放电影
\usepackage{multimedia} 


%%%%%%%%%%%%%%%%%%%%%%%%%%%%%%%%%%%%%%%%%%%%%%%%%%%%%%%%%%%%
%%%%    格式
%%%%

% 默认全屏播放
% \hypersetup{pdfpagemode={FullScreen}} 

%同时使用单列和多列
\usepackage{multicol} 

% 调整间距
%\usepackage{setspace} 
%\begin{spacing}{1.5}
%\tableofcontents \listoffigures \listoftables
%\end{spacing}

% 高亮代码,需要python包 pygments 
\usepackage{minted}

%双栏之间的间距
\setlength{\columnsep}{-0.05cm} 

% 设置段间距
%\setlength{\parskip}{-0.1\baselineskip} 

%去掉页面下方默认的导航条.
%\setbeamertemplate{navigation symbols}{}   

% 生成目录级数
\setcounter{tocdepth}{4} 

% 章节编号级数
\setcounter{secnumdepth}{4}

% 公式按章编号
\numberwithin{equation}{section} 

% 公式按节编号
%\numberwithin{equation}{subsection} 

% 图片按章编号
\numberwithin{figure}{section} 

% 两端对齐
\renewcommand{\raggedright}{\leftskip=0pt \rightskip=0pt plus 0cm} 
\raggedright

% 图表编号
%\setbeamertemplate{caption}[numbered] 

% 图表标题字体大小设置
%\setbeamerfont{caption}{size=\footnotesize} 

% % 调整第一页标题占位
%  \defbeamertemplate*{frametitle}{smoothbars theme}
%   {%
%     \nointerlineskip%
%     \begin{beamercolorbox}[wd=\paperwidth,leftskip=.3cm,rightskip=.3cm plus1fil,vmode]{frametitle}
%       \vskip.6ex
%       \usebeamerfont*{frametitle}\insertframetitle%
%       \vskip.6ex
%     \end{beamercolorbox}%
%   }

% % 自定义页脚
% \usefoottemplate{\hbox{\tinycolouredline{structure!80!black}{
% \color{white}{ \insertshortauthor} \hfill{\insertshortinstitute }
% \hfill{\insertframenumber\,/ \inserttotalframenumber}
% % \hfill{{\the\year}/{\the\month}/{\the\day}}
% }}}

% 页边距
\usepackage{geometry}
\geometry{left=2.0cm,right=2.0cm,top=2.0cm,bottom=2.0cm}

% 页眉页脚
\usepackage{fancyhdr}
\pagestyle{fancy}
\renewcommand{\headrulewidth}{0pt}
\lhead{}
\chead{}
\rhead{}
\lfoot{}
\cfoot{\thepage}
% 每页右下脚添加目录页的超链接,/tableofcontents \label{TABLE}
\rfoot{\hyperref[TABLE]{$\vartriangle$}}


\title{C++ 学习笔记}
\author{keepone}

\begin{document}

\maketitle
\newpage 

\tableofcontents \label{TABLE}
\newpage

%%%%%%%%%%%%%%%%%%%%%%%%%%%%%%%%%%%%%%%%%%%%%%%%%%%%%%%%%%%%%
\section{STL}
\subsection{utility}
\subsubsection{variant}


\newpage 

%%%%%%%%%%%%%%%%%%%%%%%%%%%%%%%%%%%%%%%%%%%%%%%%%%%%%%%%%%%%%
\section{SFINAE}
\subsection{简介}

考虑如下的代码:

\begin{minted}[bgcolor=bgColor_]{cpp}
// header.h
struct Bar {
    typedef double internalType;  
};

template <typename T> 
typename T::internalType foo(const T& t) { 
    cout << "foo<T>\n"; 
    return 0; 
}

// main.cpp
int main() {
    foo(Bar());
    foo(0); // << error!
}
\end{minted}

如果声明一个foo的重载函数,foo(0)就会编译通过。但是编译器在编译时,
也只是看到了包含模板的头文件,并没有找到 foo 的重载,为什么没有报错。

\mint[bgcolor=bgColor_]{cpp}{int foo(int i) { cout << "foo(int)\n"; return 0; }}

\subsection{函数重载机制}
\begin{itemize}
    \item 名称查找,将找到所有模板与非模板函数。
    \item 模板类型的推导,函数模板由实参推导而来。
    \begin{itemize}
        \item 推导模板的所有参数,包括返回类型及参数类型
        \item 当推导失败时,将当前模板从函数重载集中删除,但并不会报错
    \end{itemize}
    \item 最终的函数重载集,这里会看到所有函数。
    \item 选择最合适的函数
\end{itemize}
\subsection{使用场景}
\subsection{std::enable\_if}
\subsection{SFINAE}
\subsection{缺点}
\subsection{SNINAE替换方案}
\subsubsection{Tag Dispatching}
\subsubsection{c++17 的编译时 if}
\subsubsection{c++20 的concepts}


\subsection{io}
\subsubsection{iomainip}


%%%%%%%%%%%%%%%%%%%%%%%%%%%%%%%%%%%%%%%%%%%%%%%%%%%%%%%%%%%%%
\section{Template}








%%%%%%%%%%%%%%%%%%%%%%%%%%%%%%%%%%%%%%%%%%%%%%%%%%%%%%%%%%%%%
\section{Lambda}


%============================================================
\subsection{基本概念}

%.............
\subsubsection{Lambda 表达式的多种形式}
\begin{minted}[bgcolor = bgColor_]{cpp}
// default capture is copy 
[]()[mutable] {}

// copy capture = 
[=]() [mutable] {use(title);}
[t=title]() [mutable] { decltype(title) ... use(t);}
[title]() [mutable] { decltype(title) ... use(title);}

// reference capture &
[&]() [mutable] {use(title);} // the most useful
[&t=title]() [mutable] {use(t);}
[&title]() [mutable] {use(title);}
\end{minted}

%.............
\subsubsection{引用空悬}
    \begin{itemize}
        \item 向下(入栈)传递不会导致空悬。
        \item 向上(出栈)、多线程交叉传递可能导致空悬。
    \end{itemize}

\begin{minted}[bgcolor = bgColor_]{cpp}
// 函数 hasTitle 返回的 has_title_t 
// 被向上传递到上层栈后,
// 捕获 title 当前栈就会被销毁,title 被空悬。
auto hasTitle(){
    auto has_little_t = [&t=title](const Book& b){
        return b.title() == t;
    }
    return has_title_t;    
}
\end{minted}

%.............
\subsubsection{引用捕获与拷贝捕获}
kitten 的捕获发生在生调用时,此时 g=20,
cat 的捕获发生在声明时,此时 g=10。

\begin{minted}[bgcolor=bgColor_]{cpp}
#include <stdio.h>
int g = 10;
auto kitten = [=](){return g+1;};
auto cat = [g=g](){return g+1;};

int main(){
    g = 20;
    printf("21 11\n", kitten(), cat());
}
\end{minted}

%.............
\subsubsection{局部静态变量}
局部静态变量的行为相比于类的静态变量来理解
\begin{minted}[bgcolor=bgColor_]{cpp}
auto counter = [](){static int i; return ++i;};
auto c1 = counter;
auto c2 = counter;

std::cout << c1() << c1() << c1(); // 1 2 3
std::cout << c2() << c2() << c2(); // 4 5 6 
\end{minted}


%.............
\subsubsection{拷贝与移动}
Lambda 是否可以拷贝、是否可移动取决于它的捕获对象是否可拷贝、是否可移动。
\begin{minted}[bgcolor=bgColor_]{cpp}
// move only 
std::unique_ptr<int> prop;

// capture move only object
auto lamb = [p=std::move(prop)](){};

// 
auto lamb2 = std::move(lamb);

// error: call of implicitly-deleted copy constructor
auto lamb3 = lamb; 
\end{minted}


%============================================================
\subsection{高阶使用}

%.............
\subsubsection{各版本特性}
\begin{itemize}
    \item c++14 
    \begin{itemize}
        \item generic lambdas: pass auto argument, 
            compiler expands to a function template
        \item capture with initialiser: with this fiture you can 
            capture not only existing variables from the other scope,
            but also create new state variables for lambdas. This 
            also allowed capturing moveable only types 
    \end{itemize}
    \item c++17
    \begin{itemize}
        \item constexpr lambdas: lambdas can work in a constexpr context.
        \item capturing this improvements: capture *this by copy, avoiding
            dangling when returning the lambda from a member function or 
            store it
    \end{itemize}
    \item c++20
    \begin{itemize}
        \item template lambdas: improvements to generic lambdas which offers
            move control over the input template argument
        \item lambdas and concepts: lambdas can also work with constrained 
            auto and concepts, so they are as flexible as functors as template functions 
        \item lambdas in unevaluated contexts: can now create a map or a set 
            and use a lambda as a predicate
    \end{itemize}
\end{itemize}


%.............
\subsubsection{捕获 this}
Lambda 表达式中的 this 并不代表 lambda 表达式本身,而是代表定义它的那个
定义域。下列 Widget 中的两个 this 都代表 Widget。
\begin{minted}[bgcolor=bgColor_]{cpp}
class Widget{
    void work();
    void synchronous_foo(){
        this->work();
    }
    void asynchronous_foo(){
        fire_and_front([=](){
            this->work()
        });
    }
};
\end{minted} 

捕获 this 的多种方式

\begin{minted}[bgcolor=bgColor_]{cpp}
// copy capture
[=](){this->work();} // deprecated in c++20
[this](){this->work();}
[ptr=this](){ptr->work();}

// reference capture
[&](){this->work();}

// new in c++17
[*this](){this->work();}
[obj=*this](){obj.work();}

// by move
[obj=std::move(*this)](){obj.work();}
\end{minted} 
    

%.............
\subsubsection{捕获变参}
\begin{minted}[bgcolor=bgColor_]{cpp}
// c++14/17 
auto one = [](auto&& ... args){
    return product(std::forward<decltype(args)>(args)...);
}
// c++20 perfect capture
auto two = []<class... Ts>(Ts&&... args){
    return product(std::forward<Ts>(args)...);
}
\end{minted} 


%.............
\subsubsection{Lambda 作为函数实参}
\begin{minted}[bgcolor=bgColor_]{cpp}
class Shelf{
public:
    template<class Func>
    void for_each_book(Func f){
        for(const Book& b : books_) f(b);
    }
}

Shelf myshelf;
myshelf.for_each_book([](auto && book){book.print();});
\end{minted} 


%.............
\subsubsection{重载 Lambda}
\begin{minted}[bgcolor=bgColor_]{cpp}
#include <iostream>
#include <type_traits>
#include <functional>

template<class F1, class... Fs>
struct overload : F1, overload<Fs...>
{
    using F1::operator();
    using overload<Fs...>::operator();
    overload(F1 f1, Fs... fs) : F1(f1), overload<Fs...>(fs...) {}
};

template<class F1>
struct overload<F1> : F1
{
    using F1::operator();
    overload(F1 f1) : F1(f1) {}
};


template <class... F>
auto make_overload(F... f) {
    return overload<F...>(f...);
}

int main() {

    std::function <int(int,int)> f = [](int x,int y) {
        return x+y;
    };
    std::function <double(double,double)> g = [](double x,double y) {
        return x+y;
    };
    std::function <std::string(std::string,std::string)> h = [](std::string x,std::string y) {
        return x+y;
    };

    auto fgh = make_overload(f,g,h);
    std::cout << fgh(1,2) << std::endl;
    std::cout << fgh(1.5,2.5) << std::endl;
    std::cout << fgh("bob","larry") << std::endl;
}
\end{minted}


%.............
\subsubsection{继承Lambda}
\begin{minted}[bgcolor=bgColor_]{cpp}
template<typename TFirst, typename... TRemaining>
class FunctionSequence : public TFirst, FunctionSequence<TRemaining...>{
public:
    FunctionSequence(TFirst first, TRemaining... remaining)
        : TFirst(first), FunctionSequence<TRemaining...>(remaining...)
    {}

    template<typename... Args>
    decltype(auto) operator () (Args&&... args){
        return FunctionSequence<TRemaining...>::operator()
            (TFirst::operator()(std::forward<Arg>(args)...));
    }
};

template<typename T>
class FunctionSequence<T> : public T{
    public:
    FunctionSequence(T t) : T(t) {}
    using T::operator();
};

template<typename... T>
auto make_functionSequence(T... t){
    return FunctionSequence<T...>(t...);
}

int main(){
    //note: these lambda functions are bug ridden. Its just for simplicity here.
    //For correct version, see the one on coliru, read on.
    auto trimLeft = [](std::string& str) -> std::string& { 
            str.erase(0, str.find_first_not_of(' ')); 
            return str; 
    };
    auto trimRight = [](std::string& str) -> std::string& { 
        str.erase(str.find_last_not_of(' ')+1); 
        return str; 
    };
    auto capitalize = [](std::string& str) -> std::string& { 
        for(auto& x : str) x = std::toupper(x); 
        return str; }
    ;

    auto trimAndCapitalize = make_functionSequence(trimLeft, trimRight, capitalize);
    std::string str = " what a Hullabaloo     ";

    std::cout << "Before TrimAndCapitalize: str = \"" << str << "\"\n";
    trimAndCapitalize(str);
    std::cout << "After TrimAndCapitalize:  str = \"" << str << "\"\n";

    return 0;
}
\end{minted}


\begin{minted}[bgcolor=bgColor_]{cpp}
#include <iostream>
#include <type_traits>

template<typename Lhs, typename Rhs>
struct S : Lhs, Rhs{
    S(Lhs lhs, Rhs rhs) : Lhs(lhs), Rhs(rhs) {}
    using Lhs::operator();
    using Rhs::operator();
};

template<typename Lhs, typename Rhs>
auto make_combined(Lhs &&lhs, Rhs &&rhs){
    return S<std::decay_t<Lhs>, std::decay_t<Rhs>>(
        std::forward<Lhs>(lhs), std::forward<Rhs>(rhs)
    );
}

int main(){
    auto l1 = [](){return 3;};
    auto l2 = [](int i){return i*10;};
    auto combined = make_combined(l1, l2);
    std::cout << combined() << std::endl;
    std::cout << combined(3) << std::endl;
    return 0;
}
\end{minted}

%============================================================
\subsection{Lambda vs Closure(闭包)}

闭包是编程的一个概念,来自于函数式编程,c++中的 Lambda 表达式就是一种闭包。

\mint[bgcolor=bgColor_]{cpp}{auto f = [&](int x, int y) {return x+y;};}

\begin{itemize}
    \item c++ Lambda是一个表达式,该表达式被编译后生成一个闭包以在运行时工作。
    \item lambda 与闭包类似于类与类的实现,前者只存在于代码中,不存在于运行时。
    \item 上式中的 f 并不是闭包,而是闭包的拷贝,闭包是一个临时量。如果要保存闭包,可以使用
        万能引用实现,如下图所示。
\end{itemize}

\mint[bgcolor=bgColor_]{cpp}{auto f&& = [&](int x, int y) {return x+y;};}


%============================================================
\subsection{Lambda vs std::function}

%.............
\subsubsection{异同}
\begin{itemize}
    \item std::function 只能存储可拷贝对象,lambda 可存储任意对象
    \item std::function 永远可以拷贝,lambda 的拷贝与移动性取决于其捕获对象的拷贝、移动性
\end{itemize}

%.............
\subsubsection{传递成员函数}

\end{document}