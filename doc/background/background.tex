
%%%%%%%%%%%%%%%%%%%%%%%%%%%%%%%%%%%%%%%%%%%%%%%%%%%%%%%%%%%%
%%%%   表格 
%%%%
%\usepackage{colortbl} % 彩色表格
%\usepackage[table]{xcolor} % 自动加载 colortbl 为表格染色
%\usepackage{booktabs} % 可以使用三线表
\usepackage{multirow} % 复杂表格,使用multirow必须加载该package
%\usepackage{diagbox} % 有斜线表头的表格
%\usepackage{dcolumn} % 表格中小数点对齐


%%%%%%%%%%%%%%%%%%%%%%%%%%%%%%%%%%%%%%%%%%%%%%%%%%%%%%%%%%%%
%%%%   图片
%%%%
\usepackage{graphicx,subfigure,boxedminipage}
\usepackage{tikz}
\usepackage{overpic}
% 浮动体
\usepackage{float}

%图片所在的目录
\graphicspath {{figure/}} 

% 校徽logo
%\logo{\includegraphics[height=0.1\textwidth]{hfut_logo_red.pdf}} 
%\logo{\pgfuseimage{logo}}
%\pgfdeclareimage[height=5cm]{logo}{ustc_logo_new.pdf}


%%%%%%%%%%%%%%%%%%%%%%%%%%%%%%%%%%%%%%%%%%%%%%%%%%%%%%%%%%%%
%%%%   颜色
%%%%
\usepackage{xcolor}
%\usepackage{color}
%% beamer中已经定义的颜色:
%% red,green,blue,cyan,magenty,yellow,black,
%% darkgray,gray,lightgray,orange,violet,purple,brown
%% 自定义颜色:
%% \xdefinecolor{lanvendar}{rgb}{0.8,0.6,1}
%% \xdefinecolor{olive}{cmyk}{0.64,0,0.95,0.4}
%% \colorlet{structure}{blue!60!black}
% 自定义颜色,用“structure”表示 60%蓝色+40%黑色的颜色
\colorlet{structure}{blue!85!white}

%自定义背景颜色
\definecolor{bgColor_}{rgb}{0.8,0.85,0.85}

% 背景色, 上25%的蓝, 过渡到下白.
%\setbeamertemplate{background canvas}[vertical shading][bottom=white,top=structure.fg!25]
%\beamertemplateshadingbackground{white}{blue!25} %设置渐变(gradient)背景色,
%\beamersetaveragebackground{yellow!25} % 设置单一的(solid)背景色
%\beamertemplategridbackground[0.3cm] % 设置栅格(grid ) 背景

% 自定义命令. item 逐步显示时, 
% 使将要出现的item、正在显示的item、已经出现的item、 呈现不同颜色.
\def\hilite<#1>{\temporal<#1>{\color{blue!80}}{\color{blue!85!white}}{\color{black}}} % magenta

% 自定义命令,文体着色加粗
\newcommand{\renderTextBold}[2][blue]{\textbf{\color{#1}#2}}

% 自定义命令,段落背景色,可以包含引用
\newcommand{\cbox}[2][bgColor_]{\colorbox{#1}{\parbox{\linewidth}{\strut#2\strut}}}

% %%%%%%%%%%%%%%%%%%%%%%%%%%%%%%%%%%%%%%%%%%%%%%%%%%%%%%%%%%%%
% %%%%    主题
% %%%%
% \mode<presentation> {

%   % With navigation bar: default, boxes, Bergen, Madrid, Pittsburgh, Rochester
%   % With a treelike navigation bar: Antibes, JuanLesPins, Montpellier.
%   % With a TOC sidebar: Berkeley, PaloAlto, Goettingen, Marburg, Hannover
%   % With a mini frame navigation: Berlin, Ilmenau, Dresden, Darmstadt, Frankfurt, Singapore, Szeged
%   % With section and subsection titles: Copenhagen, Luebeck, Malmoe, Warsaw  
%   \usetheme{Antibes} %[hideothersubsections] beamer 模板的模式

%   % Inner color themes, 其他选择: orchid,albatross,beaver,beetle,default,crane,dolphin,dove,fly,orchid,lily,rose,seagull,seahorse
%   % Inner color themes, 其他选择: sidebartab,whale,wolverine
%   \usecolortheme{sidebartab}% default

%   % default,circles,margin,rounded,rectangles
%   \useinnertheme[shadow=true]{rounded} 

%   % default,infolines,miniframes,shadow,smoothbars,smoothtree,tree,sidebar,split
%   % \useoutertheme[height=0.1\textwidth,width=0.15\textwidth,hideothersubsections]{sidebar}

%   \setbeamercovered{dynamic} % dynamic,transparent,invisible

% }



%%%%%%%%%%%%%%%%%%%%%%%%%%%%%%%%%%%%%%%%%%%%%%%%%%%%%%%%%%%%
%%%%   字体
%%%%

% 英文默认字体都是 Times New Roman
%\usepackage{times} 
%\usepackage[english]{babel}

% 中文
\usepackage[fontset=windowsnew]{ctex}

\setmainfont{Times New Roman}
\setCJKmainfont{KaiTi}

%\usefonttheme[onlymath]{serif}
%\usefonttheme{serif} % Times New Rome 字体
%\setCJKmainfont{SimHei}
%\setCJKmainfont{FangSong}
%\setCJKmainfont{KaiTi}
%\setCJKmainfont{YouYuan}
%\setCJKmainfont{LiSu}
%\setCJKmainfont{Georgia}
%\setCJKsansfont{SimHei}
%\setCJKsansfont{FangSong}
%\setCJKsansfont{KaiTi}
%\setCJKsansfont{YouYuan}
%\setCJKsansfont{LiSu}
%\setCJKsansfont{Georgia}


%%%%%%%%%%%%%%%%%%%%%%%%%%%%%%%%%%%%%%%%%%%%%%%%%%%%%%%%%%%%
%%%%   其它
%%%%
\usepackage{enumerate}

% 抄录环境
\usepackage{alltt}
%\usepackage{verbatim}

% 数学符号
\usepackage{bm} %加粗版本的符号
\usepackage{amsmath,amssymb,amsfonts}

%超链接
\usepackage[colorlinks, linkcolor=red,anchorcolor=blue,citecolor=green]{hyperref} 

% 播放电影
\usepackage{multimedia} 


%%%%%%%%%%%%%%%%%%%%%%%%%%%%%%%%%%%%%%%%%%%%%%%%%%%%%%%%%%%%
%%%%    格式
%%%%

% 默认全屏播放
% \hypersetup{pdfpagemode={FullScreen}} 

%同时使用单列和多列
\usepackage{multicol} 

% 调整间距
%\usepackage{setspace} 
%\begin{spacing}{1.5}
%\tableofcontents \listoffigures \listoftables
%\end{spacing}

% 高亮代码,需要python包 pygments 
\usepackage{minted}

%双栏之间的间距
\setlength{\columnsep}{-0.05cm} 

% 设置段间距
%\setlength{\parskip}{-0.1\baselineskip} 

%去掉页面下方默认的导航条.
%\setbeamertemplate{navigation symbols}{}   

% 生成目录级数
\setcounter{tocdepth}{4} 

% 章节编号级数
\setcounter{secnumdepth}{4}

% 公式按章编号
\numberwithin{equation}{section} 

% 公式按节编号
%\numberwithin{equation}{subsection} 

% 图片按章编号
\numberwithin{figure}{section} 

% 两端对齐
\renewcommand{\raggedright}{\leftskip=0pt \rightskip=0pt plus 0cm} 
\raggedright

% 图表编号
%\setbeamertemplate{caption}[numbered] 

% 图表标题字体大小设置
%\setbeamerfont{caption}{size=\footnotesize} 

% % 调整第一页标题占位
%  \defbeamertemplate*{frametitle}{smoothbars theme}
%   {%
%     \nointerlineskip%
%     \begin{beamercolorbox}[wd=\paperwidth,leftskip=.3cm,rightskip=.3cm plus1fil,vmode]{frametitle}
%       \vskip.6ex
%       \usebeamerfont*{frametitle}\insertframetitle%
%       \vskip.6ex
%     \end{beamercolorbox}%
%   }

% % 自定义页脚
% \usefoottemplate{\hbox{\tinycolouredline{structure!80!black}{
% \color{white}{ \insertshortauthor} \hfill{\insertshortinstitute }
% \hfill{\insertframenumber\,/ \inserttotalframenumber}
% % \hfill{{\the\year}/{\the\month}/{\the\day}}
% }}}

% 页边距
\usepackage{geometry}
\geometry{left=2.0cm,right=2.0cm,top=2.0cm,bottom=2.0cm}

% 页眉页脚
\usepackage{fancyhdr}
\pagestyle{fancy}
\renewcommand{\headrulewidth}{0pt}
\lhead{}
\chead{}
\rhead{}
\lfoot{}
\cfoot{\thepage}
% 每页右下脚添加目录页的超链接,/tableofcontents \label{TABLE}
\rfoot{\hyperref[TABLE]{$\vartriangle$}}